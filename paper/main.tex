\documentclass[acmsmall]{acmart}

\setcopyright{acmcopyright}
\copyrightyear{2018}
\acmYear{2018}
\acmDOI{10.1145/1122445.1122456}

\begin{document}

\title[Towards user-friendliness in theorem provers]{Towards user-friendliness
  in proof assistants: an effect-based attempt}

\author{April Gonçalves}
\email{april@cyberglot.me}
\affiliation{
  \institution{Metastate AG}
  \city{Edinburgh}
  \state{UK}
}

\renewcommand{\shortauthors}{April Gonçalves}

\begin{abstract}
Proof assistants provide a framework for modelling and verification of
theories, as well as trust-worthy software. However, such power is usually only
available to experts. We propose a new approach, based on algebraic effects and
handlers, to integrate different automated proof strategies that enables
newcomers to take advantage of proof assistants without a more in-depth
understanding of underlying theory. Our approach gives beginner users an effect
system, a handful of effectful strategies (tactics, proof search and a SMT
solver) and their handlers under a shared interface, while advanced users can
extend our system with new effects and new handlers. Lastly, we prototype the
system as a library in Agda. While our prototype is minimal, it shows how easily
proofs can be carried on so long as the user has the correct intuition. We
believe our system empowers non-experts and has the potential to bring verified
software to many relevant industries, such as finance.
\end{abstract}

\keywords{effect handlers, proof assistants, user experience}

\maketitle

\section{Introduction}

The usability of proof assistants such as Coq, Agda, Isabelle/HOL and Twelf is
hard to measure: all of them require significant domain knowledge that
frequently matches a deeper understanding of Type Theory, the theory that
enables such assistants to exist. Coq and Isabelle/HOL are more commonly applied
outside their niche, and their programming style is sharply different from
mainstream programming and even functional programming. Granted, they come with
their own integrated development environment (IDE), which may make it easier for
students and newcomers. To our knowledge, no usability study has been
held for a proof assistant, however there exists studies that account for
an informal ``usability'' criteria -- they appear to measure how fast users can
familiarise themselves with the system or IDE.

Folklore within the proof assistants community seems to show that typical users
find Coq hard to use due to a large library of tactics and difficult readability
of the code after its completion. Many Coq learning materials suggest users to
add comments for structural induction cases and inductive hypothesis. As
mentioned, no user studies were conducted to attest such hypotheses.

It also has to be granted that proof assistants are relatively new technology
and there are no guidelines or even an intuition on how such systems should
perform or what features should be available to the users. The closest model we
have are proofs by hand, however, following that model for proof assistants has
two main pitfalls: (a) proofs by hand are also not widely taught in mathematics
classes during school years; and (b) proofs by hand and automated proof
writing employ fundamentally different paradigms, where the latter requires
axioms and formulations to be encoded explicitly, commonly known as a ``pedantic
proof style''.

\subsection{Empowering users of different levels of expertise}

To popularise software verification, we mush make proof writing less of a
burden to the user. Currently, only specialists can write proofs for their
programs, even though the working programmer understands the domain and
invariants of their projects -- otherwise they wouldn’t be able to write any
software, but usually the learning curve of automated formal proofs is steep
and considered too expensive.

In our approach, we propose an (algebraic) effects and handlers view of such
proofs, based on prior work developed by the Andromeda proof assistant. Here,
our users will program as they would normally and invoke
a proof environment as an effect to prove certain properties as they go. Given
that the proof environment is ``just'' an effect, we envision that different proof
styles (e.g, agda-style dependent types, SMT solver, proof search) can be ``composed''
under a shared interface, a proof object that can manipulate itself while
querying different automated proof engines.

The reasons we employ algebraic effects and handlers are two-fold:
\begin{enumerate}
\item as proofs cannot be fully automated, all approaches that try to automate
  the process (e.g, proof search, SMT solver) may either be non-deterministic or
  never find a solution. Therefore, the system should be able to handle
  ``impure'' computations and errors. Algebraic effects and handlers have
  well-defined semantics and provide a simple interface for performing effects.
  With them, we avoid indiscriminate effects that are often error-prone and
  intricate effects machinery such as monad transformers.
\end{enumerate}

\subsection{Main contributions}

\end{document}
\endinput
